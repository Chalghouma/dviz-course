\documentclass[11pt,article,oneside]{memoir} %{{{
% based on Kieran Healy's syllabus templates
% https://github.com/kjhealy/latex-custom-kjh 

\usepackage{org-preamble-pdflatex} 
\usepackage[margin=1.2in]{geometry}

\usepackage{enumitem}
\setlist{nolistsep}

\setlength{\parskip}{10pt}
\setlength{\parindent}{0pt}

%}}}
% Definitions %{{{
\def\myauthor{Author}
\def\mytitle{Title}
\def\mycopyright{\myauthor}
\def\mykeywords{}
\def\mybibliostyle{plain}
\def\mybibliocommand{}
\def\mysubtitle{}
\def\myaffiliation{Indiana University}
\def\myaddress{\url{https://iu.zoom.us/my/yyahn}} 
\def\myemail{yyahn@iu.edu}
\def\myweb{http://yongyeol.com}
\def\myphone{}
\def\myversion{}
\def\myrevision{}

\def\myaffiliation{Indiana University}
\def\myauthor{Yong-Yeol (YY) Ahn}
\def\mykeywords{Visualization, Data, Undergraduate, Informatics}
\def\mysubtitle{Syllabus}
\def\mytitle{{\normalsize DSCI 590 (Summer 2021)} \\ \HUGE{} Data Visualization}

%%\chapterstyle{article-3}
%\pagestyle{kjh}

\def\ind{\hangindent=1 true cm\hangafter=1 \noindent}
\def\labelitemi{$\cdot$}

\chapterstyle{article-4}  % alternative styles are defined in latex-custom-kjh/needs-memoir/

%}}}
\begin{document} %{{{

\title{\LARGE \mytitle} %{{{
\author{\Large\myauthor\newline \footnotesize\texttt{\noindent\myemail} }
%\date{Info East 130 (M) / 109 (W)\newline MW 4:00pm--5:15pm. \newline Office hours: W 9:15am-10am}
\date{Office hours: Tuesday 1pm-4pm at \myaddress}

%\published{\sffamily I590/H400/I400 / Fall 2014 / Mon \& Wed 4:00--5:15pm / Info West 107 (M) \& 109 (W)}

\maketitle 

\vspace{-20pt}{\bfseries Assistant Instructor} \\ %: TBD} \\ 
TBD
%Elise Jing (\texttt{jingy@iu.edu}), Office hours: Monday 5:30pm-6:30pm and Tuesday 2pm-3pm\\

%}}}
\section{Course Description}%{{{

From news to cutting-edge scientific papers, from a home office to the largest companies in the world, data visualization is a critical step in understanding data. 
Because data visualization is indispensable in data analysis, data visualization has become an essential skill for every knowledge worker.  
This course is an introduction to basic statistical data analysis and visualization.  
We will learn the fundamentals of data visualization in the context of perception, integrity, design, statistics, types of data, and visualization techniques.  
The hands-on exercises using the Python stack aim to equip you with practical data visualization skills. 

\paragraph{Relationships with E583/Z637 Information Visualization (IVMOOC)} Compared with E583/Z637, this course is geared more towards producing fundamental statistical visualizations and exploratory data analysis by writing code using the Python data science and visualization stack.  
Therefore, this course may be more suitable for students who pursue their careers in research, development, engineering, and data analysis, and those who will directly process and analyze complex datasets. 

%}}}
\section{Course Objectives}%{{{

By the end of the course, you are expected to be able to understand, explain, and manipulate basic types of data, analyze them by applying exploratory visualization techniques, and create explanatory visualizations. 
You are also expected to be able to evaluate and improve the effectiveness of data visualizations based on the fundamental visualization principles about human perception, design, data types, and visualization techniques. 
 

%}}}
\section{Communication} %{{{

Announcements, Q\&As, and other communication will be sent via Canvas and Slack. Joining Slack is optional but highly encouraged. When joining, feel free to avoid using your full name (e.g., ``John D.''). Slack allows quick messaging between individuals and groups (it is very similar to \texttt{discord} if you have used it). The address of the course slack is: 

\url{https://iu-dviz-course.slack.com}

You can create an account by using one of the following IU email addresses: \texttt{indiana.edu}, \texttt{umail.iu.edu}, \texttt{iu.edu}, and \texttt{iupui.edu}. Visit \url{https://join.slack.com/t/iu-dviz-course/signup} to signup.

Note that Email and Canvas will be much slower because the instructors are under a constant bombardment of emails about all kinds of things that you don't even want to know. 

Whenever you are not happy about the course or have a suggestion for improving the course, please share your thoughts! You can simply send a message on slack, or anonymously share your opinion:

\url{https://forms.gle/MzzNSV6Y8deJWGC77} 

%}}}
\section{Prerequisites}%{{{
\label{sec:Prerequisites}

Because producing visualizations using Python data \& visualization stack is an integral part of the course, it is required to have a good understanding and working knowledge of programming (esp. Python), as well as working knowledge of using open-source libraries. 
It is also recommended to have basic understanding of mathematics, statistics, and Web (HTML, CSS, Javascript, and JSON). 

For self-assessment, visit the following link: \href{http://bit.ly/dvizselfassess}{http://bit.ly/dvizselfassess}. 
Contact the instructor if you are uncertain about your background. 

%}}}
\section{Expectations and Requirements}%{{{
\label{sec:requirements}

%\paragraph{(All sections)} 
%The final assessment will be mainly based on a mid-term exam and a final course project. 
The primary assessment will be through the exam and the final course project. 
Although the topic will be of your choice, I strongly encourage everyone to consult with the instructors. 
You are required to submit a final paper that contains detailed explanation of the visualization \emph{process} and \emph{results}. 

You are expected to complete all course modules (quizzes and discussions) and assignments. You are also expected to engage in discussion on Canvas and Slack. 

%\paragraph{(Residential course)} You are expected to attend every class and engage in class discussions. 
%You are not allowed to use your phone or computer during the class unless explicitly asked to do so.  
%You are expected to read assigned reading materials prior to the class meetings if there is any.
%At the beginning of most class meetings, there will be an \emph{in-class quiz} based on the assigned readings and materials from the previous classes. 
%You are expected to complete all weekly assignments. 

%\paragraph{(Online)} You are expected to complete all course modules and assignments. 
%You are also expected to engage in discussions on Canvas and Slack. 

%}}}
\section{Grading}\label{sec:grading_tentative_}%{{{

The grade may be curved at the end of the course. Therefore, \emph{the grade that you can see on Canvas may not be the final grade}. 

There will be extra credits based on your strong engagement in the course, in terms of sharing useful resources \& interesting visualization-related articles, participating in discussions, and helping other students.

\begin{itemize}%{{{

\item Attendance, Quiz, and Participation: 20\%
%\item Attendance, Quiz, and Participation: 30\% 

\item Assignments: 20\% 

\item Exam: 30\%

\item Final project: 30\%
%\item Final project: 40\%

\end{itemize}%}}}
%}}}
\section{Books and key materials}%{{{

There is no required textbook, but we will mainly use materials from the following books:

\begin{enumerate}
    
\item \href{https://serialmentor.com/dataviz/}{Fundamentals of Data Visualization} by Claus O. Wilke (available online at \url{https://serialmentor.com/dataviz/})

\item \href{http://www.amazon.com/gp/product/0961392142}{The Visual Display of Quantitative Information (2nd ed.)} by E.R. Tufte: one of the foundational book on visualization. It contains a rich set of historical visualization, thoughtful discussion on visualization principles. 

\end{enumerate}

See also \href{}{Visualization books} on my wiki (\url{http://yyahnwiki.appspot.com/Information_visualization}). 


If you are still in the process of learning the basics of Python, the following books and websites may be helpful for you:

\begin{enumerate}%{{{

\item \url{https://docs.python.org/3/}: Python 3 Official Documentation

\item \url{http://www.diveintopython3.net/index.html}: Dive Into Python by Mark Pilgrim 

\item \url{http://www.learnpython.org}: A web-based interactive tutorial 

\item \url{http://ipython.rossant.net}: Learning IPython for Interactive Computing and Data Visualization by Cyrille Rossant: Introduction to IPython as well as lots of advanced analysis 

\end{enumerate}%}}}
%}}}
%}}}
%}}}

%}}}
\section{Policies}%{{{
\begin{enumerate}%{{{

\item \emph{Be honest.} Your assignments and papers should be your own work.  
If you find useful resources for your assignments, share them and cite them. 
If your friends helped you, acknowledge them. 
You should feel free to discuss both online and offline, but do not show your code directly.  
Any cases of academic misconduct (cheating, fabrication, plagiarism, etc) will be reported to the School and the Dean of Students, following the standard procedure. 
\emph{Cheating is not cool}. 

\item \emph{You have the responsibility of backing up all your data and code}.
Always back up your code and data. You should at least use Box, Dropbox, or Google Drive at the minimum.
Ideally, learn version control systems and use \url{https://github.iu.edu} or \url{https://github.com}. 
Loss of data, code, or papers (e.g.~due to malfunction of your laptop) is not an acceptable excuse for delayed or missing submission. 

\item \emph{Disabilities.} Every attempt will be made to accommodate qualified
students with disabilities (e.g.~mental health, learning, chronic health,
physical, hearing, vision, neurological, etc.). You must have established your
eligibility for support services through Disability Services for Students. Note
that services are confidential, may take time to put into place, and are not
retroactive.  Captions and alternate media for print materials may take three
or more weeks to get produced. Please contact Disability Services for Students
at \url{http://disabilityservices.indiana.edu} or 812-855-7578 as soon as
possible if accommodations are needed. The office is located on the third
floor, west tower, of the Wells Library (Room W302). Walk-ins are welcome 8 AM
to 5 PM, Monday through Friday. You can also locate a variety of campus
resources for students and visitors who need assistance at
\url{http://www.iu.edu/~ada/index.shtml}. 

\item \emph{Bias-based incidents.} Any act of discrimination or harassment based on 
race, ethnicity, religious affiliation, gender, gender identity, sexual orientation, or
disability can be reported to \texttt{biasincident@indiana.edu} or to the Dean of Students Office at (812) 855-8188. 

\item \emph{Sexual misconduct and Title IX.} As your instructor, one of my
responsibilities is to create a positive learning environment for all students.
Title IX and IU's Sexual Misconduct Policy prohibit sexual misconduct in any
form, including sexual harassment, sexual assault, stalking, and dating and
domestic violence.  If you have experienced sexual misconduct, or know someone
who has, the University can help. If you are seeking help and would like to
speak to someone confidentially, you can make an appointment with:

\begin{enumerate}
    
\item The Sexual Assault Crisis Services (SACS) at (812) 855-8900 (counseling services)
\item Confidential Victim Advocates (CVA) at (812) 856-2469 (advocacy and advice services)
\item IU Health Center at (812) 855-4011 (health and medical services)

\end{enumerate}

It is also important that you know that Title IX and University policy require me to share any information brought to my attention about potential sexual misconduct, with the campus Deputy Title IX Coordinator or IU's Title IX Coordinator. 
In that event, those individuals will work to ensure that appropriate measures are taken and resources are made available. 
Protecting student privacy is of utmost concern, and information will only be shared with those that need to know to ensure the University can respond and assist. 
I encourage you to visit \emph{stopsexualviolence.iu.edu} to learn more. 

%\item \emph{Bring your laptop on Wednesdays}. However, \emph{no electronics---laptops, tablets, and smartphones---may be used in the class}, unless the usage is specifically requested by the instructors. 
%It has been shown that \href{http://www.scientificamerican.com/article/a-learning-secret-don-t-take-notes-with-a-laptop/}{using laptops in class hurts learning, \emph{even if} you are using it to take notes}.  
%If you must have electronics due to a special reason, please obtain a permission beforehand. 

%\item \emph{Inform your excused absences prior to class}. Please contact the instructor prior to the class that you cannot attend. 

%\item \emph{Late assignments}. There will be 10\% late penalty for the late assignments unless excused. 

\item If you have any mental issues, don't hesitate to contact \href{http://healthcenter.indiana.edu/counseling/index.shtml}{IU's Counseling and Psychological Services}, which provides free counseling sessions. 


\end{enumerate}%}}}
%}}}
\clearpage\section{Course Schedule}%{{{

The schedule may change due to unexpected circumstances. See also \href{https://registrar.indiana.edu/official-calendar/index.shtml}{IU Official Calendar}. 

\subsection{Key dates}\label{sub:key_dates} %{{{

Mark your calendar and plan ahead!

\begin{itemize}%{{{
\item Exam: The last week of the semester%\textbf{11/18}
\item Project proposal due: \textbf{6/11}
\item Project presentation and paper due: \textbf{7/23}
\item Final exam due: \textbf{7/30}
\end{itemize} %}}}

%}}}
\subsection{Schedule and Readings}\label{sub:schedule}%{{{

\subsubsection{Week 1 (5/10-): Why visualization?} %{{{

\begin{itemize}\itemsep0em 
\item J. Heer \emph{et al}. A Tour through the Visualization Zoo. \url{https://queue.acm.org/detail.cfm?id=1805128}
\item J. VanderPlas, The Python Visualization Landscape. \url{https://youtu.be/FytuB8nFHPQ}
\item Further readings: \url{https://github.com/yy/dviz-course/blob/master/m01-intro/class.md}
\end{itemize}	

%}}}
\subsubsection{Week 2 (5/17-): History and integrity}%{{{

\begin{itemize}\itemsep0em 
\item E.R. Tufte, The Visual Display of Quantitative Information, Ch.~1--2.
\item C.O. Wilke, Fundamentals of Data Visualization Ch.~1 (\url{https://serialmentor.com/dataviz/introduction.html}). 
\item Further readings: \url{https://github.com/yy/dviz-course/blob/master/m02-history/class.md} and \url{https://github.com/yy/dviz-course/blob/master/m03-integrity/class.md}
\end{itemize}	

%\subsubsection{Week 3 (5/20-): Labor day | Web, Declarative vs. Procedural visualization }

%}}}
\subsubsection{Week 3 (5/24-): Perception}%{{{

\begin{itemize}\itemsep0em 
\item C.G. Healey, Perception in Visualization, \url{https://www.csc2.ncsu.edu/faculty/healey/PP/index.html}
\item B. Wong, Color Coding, Nature Methods (2010).
\item B. Wong, Avoiding color, Nature Methods (2011). 
\item C.O. Wilke, Fundamentals of Data Visualization Ch.~4 Color scales (\url{https://serialmentor.com/dataviz/color-basics.html}). 
\item C.O. Wilke, Fundamentals of Data Visualization Ch.~15 Common pitfalls of color use (\url{https://serialmentor.com/dataviz/color-pitfalls.html}).
\item Further readings: \url{https://github.com/yy/dviz-course/blob/master/m04-perception/class.md}
\end{itemize}	
%}}}
\subsubsection{Week 4 (5/31-): Design }%{{{

\begin{itemize}\itemsep0em 
\item B. Wong, Gestalt Principles I \& II, Nature Methods (2010). 
\item E.R. Tufte, The Visual Display of Quantitative Information, Ch.~4.
\item S. Bateman et al., Useful Junk? The Effects of Visual Embellishment on Comprehension and Memorability of Charts, CHI'10.
\item C.O. Wilke, Fundamentals of Data Visualization Ch.~18--21 (\url{https://serialmentor.com/dataviz/optimize-data-signal.html}). 
\item Further readings: \url{https://github.com/yy/dviz-course/blob/master/m05-design/class.md}
\end{itemize}	
%}}}
\subsubsection{Week 5 (6/7-): Data Types and 1-D data } %{{{

\begin{itemize}\itemsep0em 
\item H. Wickham, Tidy Data, Journal of Statistical Software, \url{https://vita.had.co.nz/papers/tidy-data.pdf}
\item C.O. Wilke, Fundamentals of Data Visualization Ch.~14 (\url{https://serialmentor.com/dataviz/overlapping-points.html}). 
\item Further readings: \url{https://github.com/yy/dviz-course/blob/master/m06-data/class.md}
\end{itemize}	
%}}}
\subsubsection{Week 6 (6/14-): Histogram and Boxplot }%{{{

\begin{itemize}\itemsep0em 
\item C.O. Wilke, Fundamentals of Data Visualization Ch.~6--7 (\url{https://serialmentor.com/dataviz/overlapping-points.html}). 
\item Further readings: \url{https://github.com/yy/dviz-course/blob/master/m07-1D/class.md} and \url{https://github.com/yy/dviz-course/blob/master/m08-histogram/class.md}
\end{itemize}	
%}}}
\subsubsection{Week 7 (6/21-): Estimation and logscale }%{{{

\begin{itemize}\itemsep0em 
\item C.O. Wilke, Fundamentals of Data Visualization Ch.~8--9 (\url{https://serialmentor.com/dataviz/overlapping-points.html}). 
\item Further readings: \url{https://github.com/yy/dviz-course/blob/master/m09-estimation/class.md}
\item Khan Academy: Logarithmic scale with Vi Hart (\url{https://www.khanacademy.org/math/algebra2/exponential-and-logarithmic-functions/logarithmic-scale}). 
\end{itemize}	
%}}}
\subsubsection{Week 8 (6/28-): High-dimensional data }%{{{

\begin{itemize}\itemsep0em 
\item C.O. Wilke, Fundamentals of Data Visualization Ch.~11 (\url{https://serialmentor.com/dataviz/visualizing-associations.html}). 
\item 3Blue1Brown, Eigenvectors and eigenvalues \url{https://www.youtube.com/watch?v=PFDu9oVAE-g}. 
\item Victor Powell, PCA \url{http://setosa.io/ev/principal-component-analysis/}.
\item L. van der Maaten \& G. Hinton, Visualizing data using t-SNE, JMLR 2008 \url{http://www.jmlr.org/papers/volume9/vandermaaten08a/vandermaaten08a.pdf}.
\item Further readings: \url{https://github.com/yy/dviz-course/blob/master/m10-logscale/class.md} and \url{https://github.com/yy/dviz-course/blob/master/m11-highdim/class.md}
\end{itemize}	
%}}}
%\subsubsection{Week 9 (7/6-): High-dimensional data }%{{{

%\begin{itemize}\itemsep0em 
%\item C.O. Wilke, Fundamentals of Data Visualization Ch.~11 (\url{https://serialmentor.com/dataviz/visualizing-associations.html}). 
%\item 3Blue1Brown, Eigenvectors and eigenvalues \url{https://www.youtube.com/watch?v=PFDu9oVAE-g}. 
%\item Victor Powell, PCA \url{http://setosa.io/ev/principal-component-analysis/}.
%\item L. van der Maaten \& G. Hinton, Visualizing data using t-SNE, JMLR 2008 \url{http://www.jmlr.org/papers/volume9/vandermaaten08a/vandermaaten08a.pdf}.
%\end{itemize}	

%\subsubsection{Week 10 (7/8-): Mid-term}
%}}}
\subsubsection{Week 9 (7/5-): Maps }%{{{

\begin{itemize}\itemsep0em 
\item Vsauce, What does earth look like? \url{https://youtu.be/2lR7s1Y6Zig}
\item Vox, Why all world maps are wrong \url{https://youtu.be/kIID5FDi2JQ}
\item Further readings: \url{https://github.com/yy/dviz-course/blob/master/m12-maps/class.md}
\end{itemize}	
%}}}
\subsubsection{Week 10 (7/12-): Text and Networks } %{{{

\begin{itemize}\itemsep0em 
\item J. Harris, Word clouds considered harmful, \url{http://www.niemanlab.org/2011/10/word-clouds-considered-harmful/}. 
\item The Observatory of Economic Complexity, \url{https://atlas.media.mit.edu/en/profile/country/usa/}.
\item Further readings: \url{https://github.com/yy/dviz-course/blob/master/m13-text/class.md} and \url{https://github.com/yy/dviz-course/blob/master/m14-networks-and-interactive/class-network.md}
\end{itemize}	
%\subsubsection{Week 14 (11/19-): Thanksgiving break}
%\subsubsection{Week 15 (11/26-): Project Hacks}
%}}}
\subsubsection{Week 11 (7/19-): Project presentation} %{{{
%}}}
%\subsubsection{Week 14 (11/25-): Thanksgiving break} %{{{
%}}}
%\subsubsection{Week 15 (12/2-): Project Hack week}%{{{
%}}}
\subsubsection{Week 12 (7/26-): Final exam week}%{{{
%}}}

%}}}

%}}}

\end{document} %}}}
